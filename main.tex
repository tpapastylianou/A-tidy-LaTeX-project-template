\documentclass{dependencies/styles/IEEEtran}

% Import required packages, configurations, and defined macros.
%% Load required packages
\usepackage[breaklinks,colorlinks,bookmarksopen,bookmarksnumbered,linkcolor=ICSTblue,citecolor=blue,urlcolor=ICSTblue]{hyperref}
\usepackage{graphicx}
\usepackage{amsmath}
\usepackage{amssymb}
\usepackage{amsfonts}
\usepackage{cite}
\usepackage{color}
\usepackage{soul}
\usepackage[usenames,dvipsnames,svgnames,table]{xcolor}

%%  Initialise Packages that need it
\definecolor{black}{rgb}{0,0,0}
\definecolor{crayonblue}{rgb}{0,0.65,1}
\definecolor{darkcyan}{rgb}{0.0,0.6,0.6}
\definecolor{darkgreen}{rgb}{0,0.5,0}
\definecolor{darkorange}{rgb}{0.75, 0.375, 0}
\definecolor{darkpurple}{rgb}{0.5,0.25,0.75}
\definecolor{darkred}{rgb}{0.75,0,0}
\definecolor{darkviolet}{rgb}{0.6,0.2,0.4}
\definecolor{lightblue}{rgb}{0.85,0.85,1}
\definecolor{lightcrayonblue}{rgb}{0.2,0.85,1}
\definecolor{lightcyan}{rgb}{0.65,1,1}  
\definecolor{lightgreen}{rgb}{0.65,1,0.65}
\definecolor{lightlightgray}{rgb}{0.9,0.9,0.9}
\definecolor{lightlightgreen}{rgb}{0.85,1,0.85}
\definecolor{lightorange}{rgb}{1, 0.8, 0.5}
\definecolor{lightpurple}{rgb}{1,0.75,1}
\definecolor{lightred}{rgb}{1,0.65,0.65}
\definecolor{lightviolet}{rgb}{1,0.6,0.8}
\definecolor{middlegray}{rgb}{0.6,0.6,0.6}
\definecolor{white}{rgb}{1,1,1}
\soulregister\cite7
\soulregister\ref7
\soulregister\pageref7
\hypersetup{backref=true,colorlinks=true,linkcolor=red,citecolor=green,urlcolor=blue}
\hyphenation{op-tical net-works semi-conduc-tor}

%%%%%%%%%%%%%%%%%%%%%%%%%%%%%%%%%%%%%%%%%%%%%%%%%%%%%%%%%%%%%%%%%%%%%%%%%%%%%%%%%%%%%%%%%%%%%%%%%%%%
% OVERVIEW OF COMMANDS AND MACROS DEFINED IN THIS FILE
%
% Colour text markers allowing optional comments, for editing purposes; should not appear in final article
%    \cn        : '[citation(s) needed]' marker
%    \todo      : 'TODO' marker
%    \TODO      : `TODO' marker, Big block version; helpful in denoting whole section
%    \fixme     : 'FIXME' marker
%    \rephrase  :  `REPHRASE' marker
%    \tbc       :  'TO BE COMPLETED' marker
%    \confirm   :  'CONFIRM' marker
%    \consider  :  `CONSIDER' marker (typically only useful with an optional argument provided)
%
% Formatting markers, used to denote different editing contexts
%    \markChanges  :  Highlights with light blue to indicated where changes have occured (for revision purposes)
%    \markAccepted :  Highlights with light green to indicate changes that were 'accepted' (for confirmation) since last review.
%    \del          :  Colours excerpt in a light gray colour, for marking text as a temporary placeholder, or more generally marked for deletion.
%    \notes        :  Typeface active section in calligraphic font to mark as notes; useful when used as a scratchpad at end of document
%
% Togglable markers, used for revision purposes. Toggled with \correctionstrue or \correctionsfalse as appropriate in main document
%    \mccorrect     :  Highlights with light blue to indicated where changes have occured (for revision purposes)
%    \mccontext     :  Provide context for a passage, e.g. for signposting corrections made; inserts a custom note at the margin
%    \rmcontext     :  Inserts a red custom note at the margin to allow marking / explaining a deletion since last review.
%    \accontext     :  Inserts a green custom note at the margin to allow marking / explaining an accepted change since last review
%    \mccorrection  :  (environment): Similar to markChanges, but highlights an entire block.
%
% Other miscellaneous shorcuts and utilities
%    \abbr             :  Tooltip and reference abbreviations appropriately
%    \etal             :  Shortcut to type et al. in italics
%    \citeWithTooltip  :  Citations with added mouse-hover tooltip functionality, to allow easy inspection of cited paper at the point of insertion
%    \ccite            :  Quick shortcut to the \citeWithTooltip one above, making citations display their bibtex key on mouse hovering
%
%%%%%%%%%%%%%%%%%%%%%%%%%%%%%%%%%%%%%%%%%%%%%%%%%%%%%%%%%%%%%%%%%%%%%%%%%%%%%%%%%%%%%%%%%%%%%%%%%%%%


% Mark / highlight changes from last revision.
  \newcommand{\markChanges}[1]{{\sethlcolor{lightblue}\hl{#1}}}
  \newcommand{\markAccepted}[1]{{\sethlcolor{lightgreen}\hl{#1}}}

% '[citation needed]' marker with optional comment; for editing purposes, should not appear in final article
  \newcommand{\cn}[1][]{{\sethlcolor{crayonblue}\color{white}\hl{\textbf{[citation(s)~needed]}}}{\sethlcolor{lightcrayonblue}\color{darkgray}\hl{#1}}}
  
% 'TODO' marker with optional comment; for editing purposes, should not appear in final article
  \newcommand{\todo}[1][]{{\sethlcolor{orange}\color{white}\hl{\textbf{[TODO]}}}{\sethlcolor{lightorange}\color{darkgray}\hl{#1}}}

% Big block version
  \newcommand{\TODO}[1][]{{\sethlcolor{orange}\hypersetup{urlcolor=white}\href{TODO: #1}{\begin{center}\hl{\textbf{***************************}}\\\hl{\textbf{********** TODO **********}}\\\hl{\textbf{***************************}}\end{center}~}}}

% 'FIXME' marker with optional comment; for editing purposes, should not appear in final article
  \newcommand{\fixme}[1][]{{\sethlcolor{red}\color{white}\hl{\textbf{[FIXME]}}}{\sethlcolor{lightred}\color{darkgray}\hl{#1}}}

% 'REPHRASE' marker with optional comment; for editing purposes, should not appear in final article
  \newcommand{\rephrase}[1][]{{\sethlcolor{darkcyan}\color{white}\hl{\textbf{[REPHRASE?]}}}{\sethlcolor {lightcyan}\color{darkgray}\hl{#1}}}

% 'TO BE COMPLETED' marker with optional comment; for editing purposes, should not appear in final article
  \newcommand{\tbc}[1][]{{\sethlcolor{darkpurple}\color{white}\hl{\textbf{[TO BE COMPLETED... ]}}}{\sethlcolor {lightpurple}\color{darkgray}\hl{#1}}}

% 'CONFIRM' marker with optional comment; for editing purposes, should not appear in final article
  \newcommand{\confirm}[1][]{{\sethlcolor{darkviolet}\color{white}\hl{\textbf{[CONFIRM!]}}}{\sethlcolor {lightviolet}\color{darkgray}\hl{#1}}}

% `CONSIDER' marker
  \newcommand{\consider}[1][]{{\sethlcolor{darkgreen}\color{white}\hl{\textbf{[CONSIDER:]}}}{\sethlcolor {lightgreen}\color{darkgray}\hl{#1}}}

% Citations with added hover tooltip
  \newcommand{\citeWithTooltip}[2][(unspecified)]{\href{citation: #1}{\cite{#2}}}
  \newcommand{\ccite}[1]{\citeWithTooltip[#1]{#1}} % quick shortcut to show bibtex key

% Tooltip and reference abbreviations appropriately
  \newcommand{\abbr}[2]{{\hypersetup{urlcolor=black}\href{abbr: #2}{\dotuline{#1}}}}

% Shortcut to type et al. in italics
  \newcommand{\etal}{\emph{et al.}}

% Typeface active section in calligraphic font to mark as notes
  \newcommand{\notes}{\fontfamily{pzc}\fontsize{13pt}{1em}\selectfont\color{gray}} 

% Colour marked-for-deletion excerpt in a light gray colour. These will be considered for deletion in case of too much text
  \newcommand{\del}[1]{{\color{cyan}#1}} 

% Togglable corrections. Use \correctionstrue in document to enable
  \newif\ifcorrections
  \correctionsfalse
  \colorlet{shadecolor}{blue!20}
%  \OuterFrameSep=-9pt
  \newenvironment{mccorrection}
  {\ifcorrections\begin{shaded}\fi}
  {\ifcorrections\end{shaded}\fi}
  \sethlcolor{shadecolor}
  \newcommand{\mccorrect}[1]{\ifcorrections\hl{#1}\else#1\fi}

% Provide context, e.g. for corrections; inserts a custom note at the margin next to the highlighted box. Place directly
% before the intended passage (a page break may also be needed before an environment
%\reversemarginpar % required once to shift the margin location to the wide margin
  \marginparwidth=3.75em
  \newcommand{\mccontext}[1]{\ifcorrections\marginpar{\vspace{0.25em}\fontfamily{pzc}\fontsize{0.9em}{0.9em}\selectfont\parbox{4em}{\begin{shaded}\hyphenpenalty=0 #1\end{shaded} \par}}\fi}
  \newcommand{\rmcontext}[1]{\ifcorrections\marginpar{\vspace{0.25em}\fontfamily{pzc}\fontsize{0.9em}{0.9em}\selectfont\parbox{4em}{\colorlet{shadecolor}{lightred}\begin{shaded}\hyphenpenalty=0 #1\end{shaded} \par}}\fi}
  \newcommand{\accontext}[1]{\ifcorrections\marginpar{\vspace{0.25em}\fontfamily{pzc}\fontsize{0.9em}{0.9em}\selectfont\parbox{4em}{\colorlet{shadecolor}{lightgreen}\begin{shaded}\hyphenpenalty=0 #1\end{shaded} \par}}\fi}



%
% --- Main document ---
%

\title { A tidy latex project template }
\author{ Tasos Papastylianou }

\begin{document}
  
  \maketitle
  
  \begin{abstract}
    This is an example document which makes use of the tidy latex project template.
  \end{abstract}

  \section{Introduction}

    Working with, and compiling \LaTeX documents can get very messy. Most people tend to dump all their files in a single folder, which then gets even more messy as the pdflatex command generates a variety of intermediate files required by the pdflatex compiler.

    This project template separates the various components of a latex project into their respective folders in a tidy manner, while also providing an example compilation script (for bash-supporting systems) that can compile everything as appropriate and then clean up any intermediate files.

    Each folder contains an example file and a README.md file containing information on how to use that folder. Feel free to explore.

  \section{Example use}

    This short sample document contains examples of how one might use this template, to link to externally defined equations, tables and figures.
    
    \subsection{ Equations }

      Display equations, by creating an 'equation' environment in this main document, and within it, redirect to the relevant equation file
      containing only the actual definition of your mathematical equation:

      \begin{equation}
         \label{eq:ExampeEquation}
         p( x ) = \frac { 1 } { \int _ a ^ b f( x ) \mathrm d x } f( x )

      \end{equation}

    \subsection{ Figures }

      Display figures by creating your 'floating' environment (e.g. via the graphicx package's includefigure command), and importing your figure file directly within it:

      \begin{figure}[h]
        \centering
        \includegraphics[width=1\linewidth]{resources/figures/duckbunny}
        \caption{A duck. Or is it a bunny?}
        \label{fig:duckbunny}
      \end{figure}

    \subsection{ Tables }

      Place tables in your document by creating your 'table' environment, and importing your table file (a normal .tex file containing a 'tabular' definition) directly within:

    \begin{table}[h]
      \caption{An example table}
      \centering
      \begin{tabular}{cccc}
  \hline
  {\bf Parameter} & {\bf mean}  & {\bf se\_mean} & {\bf sd}\\
  \hline
  $a_{01}$        &     0.0863  &        0.0029  &  0.0871  \\
  $a_{02}$        &     0.0032  &        0.0061  &  0.1497  \\
  $a_{03}$        &     0.1771  &        0.0144  &  0.3475  \\
  $a_{04}$        &     0.0006  &        0.0027  &  0.0855  \\
  $a_{05}$        &     0.1319  &        0.0054  &  0.1451  \\
  $a_{11}$        &     0.0536  &        0.0041  &  0.1454  \\
  $a_{12}$        &    -0.0532  &        0.0079  &  0.2459  \\
  $a_{13}$        &     0.5468  &        0.0124  &  0.3718  \\
  $a_{14}$        &     0.5410  &        0.0056  &  0.1582  \\
  $a_{15}$        &    -0.0049  &        0.0079  &  0.2330  \\
  $a_{21}$        &     0.0718  &        0.0046  &  0.1517  \\
  $a_{22}$        &    -0.2019  &        0.0107  &  0.2570  \\
  $a_{23}$        &     0.3631  &        0.0119  &  0.3585  \\
  $a_{24}$        &    -0.1932  &        0.0049  &  0.1449  \\
  $a_{25}$        &    -0.1025  &        0.0065  &  0.1899  \\
  $S_{x_{00}}$    &     0.2353  &        0.0032  &  0.0917  \\
  $S_{x_{11}}$    &     1.3516  &        0.0138  &  0.4259  \\
  $S_{x_{22}}$    &     0.5590  &        0.0118  &  0.3038  \\
  $\xi_1$         &     1.0116  &        0.0381  &  1.0741  \\
  $\xi_2$         &     1.2758  &        0.0478  &  1.5653  \\
  $\xi_3$         &     0.7028  &        0.0703  &  1.8407  \\
  \hline
\end{tabular}

      \label{tab:ExampleTable}
    \end{table}

    \subsection{ Bibliography and Citations }

      Use citations as normal (\cite{Papastylianou2021}). Link to your bibliography file(s) directly in the bibliography section.

  %
  % ---- Bibliography ----
  %

  \bibliographystyle{dependencies/styles/IEEEtran}
  \bibliography{resources/bibliography/ExampleBibliography}

\end{document}
