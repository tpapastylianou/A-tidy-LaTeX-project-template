%%%%%%%%%%%%%%%%%%%%%%%%%%%%%%%%%%%%%%%%%%%%%%%%%%%%%%%%%%%%%%%%%%%%%%%%%%%%%%%%%%%%%%%%%%%%%%%%%%%%
% OVERVIEW OF COMMANDS AND MACROS DEFINED IN THIS FILE
%
% Colour text markers allowing optional comments, for editing purposes; should not appear in final article
%    \cn        : '[citation(s) needed]' marker
%    \todo      : 'TODO' marker
%    \TODO      : `TODO' marker, Big block version; helpful in denoting whole section
%    \fixme     : 'FIXME' marker
%    \rephrase  :  `REPHRASE' marker
%    \tbc       :  'TO BE COMPLETED' marker
%    \confirm   :  'CONFIRM' marker
%    \consider  :  `CONSIDER' marker (typically only useful with an optional argument provided)
%
% Formatting markers, used to denote different editing contexts
%    \markChanges  :  Highlights with light blue to indicated where changes have occured (for revision purposes)
%    \markAccepted :  Highlights with light green to indicate changes that were 'accepted' (for confirmation) since last review.
%    \del          :  Colours excerpt in a light gray colour, for marking text as a temporary placeholder, or more generally marked for deletion.
%    \notes        :  Typeface active section in calligraphic font to mark as notes; useful when used as a scratchpad at end of document
%
% Togglable markers, used for revision purposes. Toggled with \correctionstrue or \correctionsfalse as appropriate in main document
%    \mccorrect     :  Highlights with light blue to indicated where changes have occured (for revision purposes)
%    \mccontext     :  Provide context for a passage, e.g. for signposting corrections made; inserts a custom note at the margin
%    \rmcontext     :  Inserts a red custom note at the margin to allow marking / explaining a deletion since last review.
%    \accontext     :  Inserts a green custom note at the margin to allow marking / explaining an accepted change since last review
%    \mccorrection  :  (environment): Similar to markChanges, but highlights an entire block.
%
% Other miscellaneous shorcuts and utilities
%    \abbr             :  Tooltip and reference abbreviations appropriately
%    \etal             :  Shortcut to type et al. in italics
%    \citeWithTooltip  :  Citations with added mouse-hover tooltip functionality, to allow easy inspection of cited paper at the point of insertion
%    \ccite            :  Quick shortcut to the \citeWithTooltip one above, making citations display their bibtex key on mouse hovering
%
%%%%%%%%%%%%%%%%%%%%%%%%%%%%%%%%%%%%%%%%%%%%%%%%%%%%%%%%%%%%%%%%%%%%%%%%%%%%%%%%%%%%%%%%%%%%%%%%%%%%


% Mark / highlight changes from last revision.
  \newcommand{\markChanges}[1]{{\sethlcolor{lightblue}\hl{#1}}}
  \newcommand{\markAccepted}[1]{{\sethlcolor{lightgreen}\hl{#1}}}

% '[citation needed]' marker with optional comment; for editing purposes, should not appear in final article
  \newcommand{\cn}[1][]{{\sethlcolor{crayonblue}\color{white}\hl{\textbf{[citation(s)~needed]}}}{\sethlcolor{lightcrayonblue}\color{darkgray}\hl{#1}}}
  
% 'TODO' marker with optional comment; for editing purposes, should not appear in final article
  \newcommand{\todo}[1][]{{\sethlcolor{orange}\color{white}\hl{\textbf{[TODO]}}}{\sethlcolor{lightorange}\color{darkgray}\hl{#1}}}

% Big block version
  \newcommand{\TODO}[1][]{{\sethlcolor{orange}\hypersetup{urlcolor=white}\href{TODO: #1}{\begin{center}\hl{\textbf{***************************}}\\\hl{\textbf{********** TODO **********}}\\\hl{\textbf{***************************}}\end{center}~}}}

% 'FIXME' marker with optional comment; for editing purposes, should not appear in final article
  \newcommand{\fixme}[1][]{{\sethlcolor{red}\color{white}\hl{\textbf{[FIXME]}}}{\sethlcolor{lightred}\color{darkgray}\hl{#1}}}

% 'REPHRASE' marker with optional comment; for editing purposes, should not appear in final article
  \newcommand{\rephrase}[1][]{{\sethlcolor{darkcyan}\color{white}\hl{\textbf{[REPHRASE?]}}}{\sethlcolor {lightcyan}\color{darkgray}\hl{#1}}}

% 'TO BE COMPLETED' marker with optional comment; for editing purposes, should not appear in final article
  \newcommand{\tbc}[1][]{{\sethlcolor{darkpurple}\color{white}\hl{\textbf{[TO BE COMPLETED... ]}}}{\sethlcolor {lightpurple}\color{darkgray}\hl{#1}}}

% 'CONFIRM' marker with optional comment; for editing purposes, should not appear in final article
  \newcommand{\confirm}[1][]{{\sethlcolor{darkviolet}\color{white}\hl{\textbf{[CONFIRM!]}}}{\sethlcolor {lightviolet}\color{darkgray}\hl{#1}}}

% `CONSIDER' marker
  \newcommand{\consider}[1][]{{\sethlcolor{darkgreen}\color{white}\hl{\textbf{[CONSIDER:]}}}{\sethlcolor {lightgreen}\color{darkgray}\hl{#1}}}

% Citations with added hover tooltip
  \newcommand{\citeWithTooltip}[2][(unspecified)]{\href{citation: #1}{\cite{#2}}}
  \newcommand{\ccite}[1]{\citeWithTooltip[#1]{#1}} % quick shortcut to show bibtex key

% Tooltip and reference abbreviations appropriately
  \newcommand{\abbr}[2]{{\hypersetup{urlcolor=black}\href{abbr: #2}{\dotuline{#1}}}}

% Shortcut to type et al. in italics
  \newcommand{\etal}{\emph{et al.}}

% Typeface active section in calligraphic font to mark as notes
  \newcommand{\notes}{\fontfamily{pzc}\fontsize{13pt}{1em}\selectfont\color{gray}} 

% Colour marked-for-deletion excerpt in a light gray colour. These will be considered for deletion in case of too much text
  \newcommand{\del}[1]{{\color{cyan}#1}} 

% Togglable corrections. Use \correctionstrue in document to enable
  \newif\ifcorrections
  \correctionsfalse
  \colorlet{shadecolor}{blue!20}
%  \OuterFrameSep=-9pt
  \newenvironment{mccorrection}
  {\ifcorrections\begin{shaded}\fi}
  {\ifcorrections\end{shaded}\fi}
  \sethlcolor{shadecolor}
  \newcommand{\mccorrect}[1]{\ifcorrections\hl{#1}\else#1\fi}

% Provide context, e.g. for corrections; inserts a custom note at the margin next to the highlighted box. Place directly
% before the intended passage (a page break may also be needed before an environment
%\reversemarginpar % required once to shift the margin location to the wide margin
  \marginparwidth=3.75em
  \newcommand{\mccontext}[1]{\ifcorrections\marginpar{\vspace{0.25em}\fontfamily{pzc}\fontsize{0.9em}{0.9em}\selectfont\parbox{4em}{\begin{shaded}\hyphenpenalty=0 #1\end{shaded} \par}}\fi}
  \newcommand{\rmcontext}[1]{\ifcorrections\marginpar{\vspace{0.25em}\fontfamily{pzc}\fontsize{0.9em}{0.9em}\selectfont\parbox{4em}{\colorlet{shadecolor}{lightred}\begin{shaded}\hyphenpenalty=0 #1\end{shaded} \par}}\fi}
  \newcommand{\accontext}[1]{\ifcorrections\marginpar{\vspace{0.25em}\fontfamily{pzc}\fontsize{0.9em}{0.9em}\selectfont\parbox{4em}{\colorlet{shadecolor}{lightgreen}\begin{shaded}\hyphenpenalty=0 #1\end{shaded} \par}}\fi}
